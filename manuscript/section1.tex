\section{Introduction}
\label{sect:intro}  % \label{} allows reference to this 
\subsection{Astrophysical Motivation}
\subsection{Survey of Design Software}
\subsection{Hybrid Propagation Physics}
\subsection{Gaussian Beamlet Decomposition}

Traditional diffraction modeling regimes consider the optical system to be paraxial and the electromagnetic field to be essentially scalar. 

Gaussian Beamlet Decomposition (GBD) is a ray-based method of diffraction calculation that approximates an optical field as a superposition of Gaussian beams. Gaussian beams are unique in that they can be propagated along ray paths. In their seminal paper, Harvey et al\cite{Harvey15} reviews the theory of complex ray tracing used to propagate Gaussian beams. 

The Gaussian Beam takes the form: \cite{goodman17}
\begin{equation}
	V = \frac{V_o}{q(z)}exp[ik\frac{r^2}{2q(z)}]
\end{equation}

Where $V_o$ is the amplitude, $k$ is the wavenumber, $r$ is the radial coordinate in the plane perpendicular to propagation, and $q(z)$ is the complex valued constant that describes the beam's $1/e$ field size (the "waist" $w_o$) and curvature. This constant is referred to as the \textit{complex beam parameter}.
\begin{equation}
	q(z)^{-1} = \frac{1}{R(z)}+i\frac{\lambda}{\pi w(z)^2}
\end{equation}	

$q(z)$ is a convenient expression of the Gaussian beam because it fully encapsulates the information required to describe the transverse electric field of the beam as it propagates. The real part of $q(z)$ is related to the radius of curvature ($R(z)$) of the wavefront.
\begin{equation}
	R(z) = z(1+(\frac{Z_o}{z})^2)
\end{equation}
Where $Z_o$ is the rayleigh range and $z$ is the longitudinal propagation distance.
The imaginary part of $q(z)$ is related to the beam waist radius ($w(z)$)
\begin{equation}
	w(z) = w_o\sqrt{1+(\frac{z}{Z_o})^2}
\end{equation}

In the paraxial regime $q(z)$ can be propagated using the ABCD matrices of geometrical optics.

\begin{equation}
    q(z)^{-1}_{o} = \frac{C + D/q_{i}}{A + B/q_{i}}
\end{equation}

For the generally astigmatic case, $q(z)$ is a 2x2 matrix that encodes the complex curvature in two orthogonal directions\cite{Ashcraft2020,cai_decentered_nodate}. 
