\documentclass{article}

\begin{document}

\begin{abstract}
In the pursuit of directly imaging exoplanets, the high-contrast imaging community has developed a multitude of tools to simulate the performance of coronagraphs on segmented-aperture telescopes. As the scale of the telescope increases and science cases move toward shorter wavelengths, the required physical optics propagation to optimize high-contrast imaging instruments becomes computationally prohibitive. Gaussian Beamlet Decomposition (GBD) is an alternative method of physical optics propagation that decomposes an arbitrary wavefront into paraxial rays. These rays can be propagated expeditiously using ABCD matrices, and converted into their corresponding gaussian beamlets to accurately model physical optics phenomena without the need of diffraction integrals. Previous work by Harvey\cite{Harvey15} has shown that GBD can accurately model physical optics phenomena, and Worku\cite{Worku19} has introduced truncated gaussian beamlets for increasing the GBD's accuracy for systems with hard apertures. We present an open source GBD tool in Python to model physical optics phenomena, with the goal of alleviating the computational burden of complex apertures, particularly those faced when optimizing coronagraphs for large segmented-aperture telescopes. We will show the relative accuracy of GBD with and without truncated beamlets, as well as the computational efficiency of GBD compared to current open-source physical optics propagation software for applications to space and ground based segmented aperture telescopes. 
\end{abstract}


\end{document}